% Slides for talk on hydrogen fuel cells
% given in the department on October 27, 2003.
% 
% The original slides were in Prosper.  This file contains the
% translation of the original slides to Beamer.
% 
% Rouben Rostamian <rostamian@umbc.edu>
% August 31, 2004

\documentclass[CJK]{beamer}
\usetheme{default}

\usepackage{CJKutf8} % for chinese
\usepackage{hyperref}
\hypersetup{colorlinks, linkcolor=blue, citecolor=blue,
    urlcolor=blue,
    plainpages=flase,
    pdfcreator=tex,
    bookmarksopen=true,
    pdfhighlight=/P,
%    pdfauthor={Yao Qi <qiyaoltc@gmail.com>},
%    pdfcreator={cTeX},
    pdfstartview=FitH,
    pdfpagemode=UseOutlines,%UseOutlines, %None, FullScreen, UseThumbs
}

\setbeamertemplate{frametitle}[default][center]

\begin{document}
\begin{CJK}{UTF8}{bkai}

\title{非關技術}
\subtitle{淺談如何參與社區開發}
\author{
  陳韋任 \\
  chenwj@iis.sinica.edu.tw
}
\institute{
  Computer Systems Lab, Institute of Information Science, \\
  Academia Sinica, Taiwan (R.O.C.)
}
\date{Nov. 10, 2012}

%----------- titlepage ----------------------------------------------%
\begin{frame}[plain]
  \titlepage
\end{frame}

%----------- slide --------------------------------------------------%
% 第二页,致谢 里边的都是link吧,但是看不出来,最好用别的颜色或者下划线来标明。
\begin{frame}
  \frametitle{致謝}

\begin{itemize}
  \item \href{http://blog.smartbear.com/software-quality/bid/167051/14-Ways-to-Contribute-to-Open-Source-without-Being-a-Programming-Genius-or-a-Rock-Star}{14 Ways to Contribute to Open Source}
  \item \href{http://kernelnewbies.org/mailinglistguidelines}{Mailing List Guidelines}
  \item \href{http://jeff.jones.be/technology/articles/how-to-ask-for-help-on-irc/}{How to ask for help on IRC}
  \item \href{http://www.catb.org/~esr/faqs/smart-questions.html}{How To Ask Questions The Smart Way}
  \item \textcolor{red}{與會的各位 :-)}
\end{itemize}

\end{frame}

%----------- slide --------------------------------------------------%
\begin{frame}
  \frametitle{Outline}

\begin{itemize}
  \item Attitude
  \item English Matters
  \item Mailing List
  \item IRC
  \item Case Study
\end{itemize}

\end{frame}

%----------- slide --------------------------------------------------%
\begin{frame}
  \frametitle{Attitude}

\begin{itemize}
  \item 謙卑但不自卑
  \item 自信但不自傲
  \item 學問
  \item 勇於宣傳
  \item \textcolor{red}{長期耕耘}
\end{itemize}

\end{frame}

%----------- slide --------------------------------------------------%
% 第5页,语法错误可以理解,但是还是要避免拼写错误。毕竟是有软件可以自动检查的。patch或者邮件里边有拼写错误,不是太好。
% 最后一句 “别忘了中文是我们的母语”,这个有什么意思吗?是说 “能用中文,还是用中文”? 还是别的什么意思?

\begin{frame}
  \frametitle{English Matters}

\begin{itemize}
  \item 英文是用來溝通的。
  \item 能清楚表達意思即可,不需害怕拼字或文法錯誤
  \item 避免拼字或文法錯誤是禮貌!
  \item 也別忘了中文是我們的母語
  \item 能說中文的場合,就不必都用英文,怪彆扭 :p
\end{itemize}

\end{frame}

%----------- slide --------------------------------------------------%
% 第6页,一段太长,不容易阅读。可以把这个段分成3部分。介绍,订阅和归类,发送

\begin{frame}
  \frametitle{What Is Mailing List?}

Mailing List 是郵件列表。你可以訂閱它,寄信至其電郵信箱,
所有訂閱者都將讀到你的訊息。大家透過這種方式,自由的討論
各項社群事務。

\medskip
例如:
\begin{itemize}
  \item HelloGcc Mailing List \href{http://www.freelists.org/list/hellogcc}{http://www.freelists.org/list/hellogcc}
  \item QEMU Mailing List \href{https://lists.gnu.org/mailman/listinfo/qemu-devel}{https://lists.gnu.org/mailman/listinfo/qemu-devel}
  \item LLVM Mailing List \href{http://lists.cs.uiuc.edu/mailman/listinfo/llvmdev}{http://lists.cs.uiuc.edu/mailman/listinfo/llvmdev}
\end{itemize}

\end{frame}

%----------- slide --------------------------------------------------%
\begin{frame}
  \frametitle{Subscribe Mailing List}

開發者請\textcolor{red}{務必}訂閱郵件列表,以便於和其他開發者交換意見; 一般使用者也建議訂閱,以跟蹤社群發展趨勢和各項消息。
\end{frame}

%----------- slide --------------------------------------------------%
\begin{frame}
  \frametitle{Setup Mailing List Filter}

活躍的郵件列表一天有上百封信件。訂閱郵件列表時,請設定適當過濾條件加以歸類至個別的信件夾,以方便閱讀。

\medskip
小技巧:
\href{https://www.google.com/search?q=mailing+list+filter+procmail}{Google: mailing list filter procmail}

\end{frame}

%----------- slide --------------------------------------------------%
\begin{frame}
  \frametitle{Mail The Mailing List!}

建議在發信時加上自己的簽名 (signature),其中可以列出自己的簡介或是網址,讓其他人有機會認識你。

\end{frame}

%----------- slide --------------------------------------------------%
\begin{frame}[fragile]
  \frametitle{Signature Example}

\begin{verbatim}

  ... Mail Body ...

--
Wei-Ren Chen (Your Chinese Name Here!)
Homepage: http://people.cs.nctu.edu.tw/~chenwj
\end{verbatim}

\end{frame}

%----------- slide --------------------------------------------------%
% 第7页, “明确主题”, 这个很好,有没有例子,什么是明确的主题,什么是不明确的主题。
% 这里边最好加一条 “在mail list里边,任何人都没有义务回答你的问题,所以邮件发出去以后,没有回复,这样也是正常的”

\begin{frame}
  \frametitle{Mailing List General Rules}

\begin{itemize}
  \item 明確的主題
  \item 在本文中進一步描述自己的問題,提供必要的資訊
  \item 表現出自己已試著自己解決問題,展現解決問題的誠意
  \item 當別人給的解法有效時,記得回復並致謝
  \item 請用群組回信
\end{itemize}

\end{frame}

%----------- slide --------------------------------------------------%
\begin{frame}
  \frametitle{Bad Subject Example}

\begin{itemize}
  \item Subject:
  \item Subject: Help me!
  \item Subject: Hello!
\end{itemize}

\end{frame}

%----------- slide --------------------------------------------------%
\begin{frame}
  \frametitle{What Is A Good One?}

\begin{itemize}
  \item Subject: Fail to compile LLVM on Gentoo Linux
  \item Subject: MCJIT vs JIT
\end{itemize}

\end{frame}

%----------- slide --------------------------------------------------%
% 第8页,加上”不要等着只有自己遇到问题的时候才想到mail list,平时也要多帮助别人 (在力所能及的范围内)“
\begin{frame}
  \frametitle{Further Suggestion about Mailing List}

\begin{itemize}
  \item 如果不清楚該郵件列表整體的環境,可以先潛伏觀察一陣
  \item 但不要當有永遠沉默的一群
  \item 參與 ML 上的討論
  \item 分享自己的成果及經驗
  \item 不要只有在自己遇到問題時才想到 ML,平時在能力範圍之內也可以幫助他人
\end{itemize}

\end{frame}

%----------- slide --------------------------------------------------%
\begin{frame}
  \frametitle{Should I Expect Anyone Anser Me?}

\begin{itemize}
  \item \textcolor{red}{在 ML 裡面,沒有人有義務回答你的問題}
  \item 因為大家都是志願者,你不是他老闆
\end{itemize}

\end{frame}

%----------- slide --------------------------------------------------%
% 第9页,有些时候,邮件没有回复,可能有别的原因,比如,邮件没有写清楚,或者maintainer休假了,或者大家对你的问题没有什么兴趣。
\begin{frame}
  \frametitle{What If Nobody Answer Me on the ML?}

\begin{itemize}
  \item 問題描述的不清楚,沒有人懂你
  \item 你的信件引不起大家的注意
  \item maintainer 休假了
\end{itemize}

\medskip
你可以在 IRC 上這樣問:

  Anyone would like to take a look on this thread?

\end{frame}

%----------- slide --------------------------------------------------%
\begin{frame}
  \frametitle{What Is IRC?}

\underline{\href{https://addons.mozilla.org/en-US/firefox/addon/chatzilla/}{ChatZilla}} 登入
IRC 是聊天室。你可以使用 \href{http://www.irssi.org/}{irssi} 或是 Firefox 插件
\href{https://addons.mozilla.org/en-US/firefox/addon/chatzilla/}{ChatZilla} 登入
聊天室。其性質較為隨性,你甚至可以在上面和其他人閒話家常。聊天室視窗上方通常會列出注意事項,請多加留意 。

\medskip

例如:
\begin{itemize}
  \item hellogcc at freenode
  \item llvm at oftcnet
  \item qemu at oftcnet
\end{itemize}
\end{frame}

%----------- slide --------------------------------------------------%
\begin{frame}
  \frametitle{IRC General Rules}

\begin{itemize}
  \item 只管發問
  \item 盡可能清楚描述你的問題
  \item 只問與該頻道相關的問題。若是不知道合適的頻道,也可以詢問他人該在何處發問
  \item 禮貌和耐心
\end{itemize}

\end{frame}

%----------- slide --------------------------------------------------%
\begin{frame}
  \frametitle{Other Communication Media}

\begin{itemize}
  \item Blog: HelloGcc Blog \href{http://www.hellogcc.org/}{http://www.hellogcc.org/},\textcolor{red}{長期徵稿}
  \item Forum
  \item News Group
  \item Stackoverflow
\end{itemize}

\end{frame}

%----------- slide --------------------------------------------------%
\begin{frame}
  \frametitle{One More Last Thing}

\centering \textcolor{red}{DO NOT SPAM!}

\end{frame}

%----------- slide --------------------------------------------------%
% 第14页,首先应该介绍一下maintainer在open source development里边的role是什么。一个maintainer需要做什么。
% ”提交有质量的代码“,那么什么杨的代码算有质量的?我在这里想说,我们有些时候,会有这样的想法,”maintainer会找出patch里边的错误的“,
% 所以自己对自己的代码检查并不是那么严格。我建议,如果想做到”提交有质量的代码“,就要严格要求自己,不要依赖maintainer这个保障。
% 除了”参与讨论“和提交有质量的代码,最好还可以review 别人的代码 (在自己熟悉的领域)。

\begin{frame}
  \frametitle{Case Study - How To Be A Maintainer?}

Maintainer 主要負責代碼的開發、審閱別人的 patch 、引導社區的方向和修正 bug。

\begin{itemize}
  \item 逐漸在社群裡建立信任和信用
  \begin{itemize}
    \item 時常參與社群的討論
    \item 提交有質量的代碼
    \item 在自己熟悉的領域審閱別人的代碼
    % 不要出现这样的情况,就是自己写了一个新feature,在commit以后,就不管了。这样就丢失了maintain的机会。
    \item 貢獻一些新的功能或者特性,並且積極改進它們
    % 每个社区都有自己的规则,这些都是有现有maintainer的喜好决定的,所以要去适应并且熟悉这样的规则
    % 在开始阶段 (两年以内),不要试图去改变。
    \item 熟悉社區的規則
  \end{itemize}
  \item \href{http://lists.gnu.org/archive/html/qemu-devel/2012-10/msg01361.html}{[Qemu-devel] MIPS DSP for Qemu}
\end{itemize}

\end{frame}

%----------- slide --------------------------------------------------%
\begin{frame}
  \frametitle{Case Study - What Else I Can Do?}

除了提交代碼,你還可以這樣做。
\begin{itemize}
  \item 修正註解、拼寫錯誤等增進代碼質量的補丁
  \item 編寫或翻譯文件
  \item 協助測試。例如: \href{http://www.hellogcc.org/?p=98}{LLVM 3.1 測試經驗談}
\end{itemize}

\end{frame}

%----------- slide --------------------------------------------------%
\begin{frame}
  \frametitle{The End}

\centering{Q \& A}

\end{frame}

\end{CJK}
\end{document}
